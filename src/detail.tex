\section{論文内容の具体的な説明}

論文内で提案されているUniversal Correspondence Network (UCN)は大きく以下の4つの要素技術からなる.
\begin{itemize}
  \item 全畳み込みネットワークによる特徴抽出
  \item 一致性に関するロス
  \item Hard Negative Mining
  \item 畳み込み転位ネットワーク
\end{itemize}
全畳み込みネットワークによる画像からの特徴抽出は,ImageNetによって学習済みのGoogleNetモデルにおける
特徴量を実験では使っており,それによって画像内の物体に関する特徴量を抽出している.
さらに一致性に関するロスでは,
サンプリング数$N$に対して画像$I$における位置$x$に対して特徴量$F_I(x)$と取り出すとして,
正のペアでは$s=1$, 負のペアでは$s=0$において以下のようなロスを設定する.
\begin{equation*}
  L = \frac{1}{2N} \sum_{N}^{i} \left[ s_i || F_I(x_i) - F_I(x_i') ||^2 + (1 - s_i) max(0, m - ||F_I(x) - F_I'(x_i')||)^2 \right]
\end{equation*}
第二項は負のペアに関するロスであるが,ランダムなサンプリングでは多くの場合に0となってしまうので
学習が遅いという問題が有り,そのために第二項が0にならないようなロスの大きいものだけを
逆伝播するHard Negative Miningを導入している.
畳み込み転位ネットワークは転位ネットワーク\cite{ST}を畳み込み処理における
パッチ毎の正規化のために用いている.
