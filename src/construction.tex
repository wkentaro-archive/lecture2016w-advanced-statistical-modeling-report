\section{論文の構成}

``1. Introduction''では,
画像における一致性の有効性および過去の研究に関して述べており,
そこにおける問題点として一致性の距離の学習がなされていないことを上げている.
そこでUCNという提案モデルに関して述べ,そのモデルを成す要素に関して
簡単に述べた後,過去の研究との比較を行っている.

``2. Related Works''では,
一致性,距離の学習,畳み込みネットワークと転位ネットワーク,全畳込みネットワーク
に関してそれぞれ過去の研究を引用して述べ,Introductionで述べた提案モデルの妥当性を
補強している.

``3. Universal Correspondence Network''では,
提案モデルの各要素に関して述べている.
提案モデルは全畳込みネットワークによる特徴学習と,
一致性学習ロス, Hard Negative Mining, 畳み込み転位ネットワークによりなっており,
それぞれに関して実装方法にも少し触れながら具体的に説明している.

``4. Experiments''では,
``Network''の節では
ネットワークの具体的な生成方法として学習済みモデルを使ったことや,
Hard Negative Miningや転位ネットワークの有効性を示すためにそれぞれがある場合とない場合の
両方で実験したことを述べ,
``Dataset and Metrics''の節ではデータセットと実験における精度の計測方法について述べている.
``Geometric Correspondence''の節ではKITTIデータセットにおける
位置的な一致性の正解データの作成およびその精度に関して過去の研究との比較を行い,
提案モデル(UCN)の有効性を示している.
``Semantic Correspondence''の節ではPASCAL, CUB-2011データセットにおける
意味的な一致性に関して述べ,その実験結果において転位ネットワークの有効性を示している.
``Camera Motion Estimation''では位置的な一致性においてカメラが大きく動く場合にも適用可能であること
をKITTIデータセットにおいてデータを変形させ実験し,有効性を示している.

``5. Conclusion''では,
提案モデルの各要素技術である
一致ロス,Hard Negative Mining, 畳み込み転位ネットワーク
について触れ,その有効性が実験において示されたことに関して述べている.
また,将来的な展望としてUCNのアプリケーションと大域的な最適化を含めた実験について
行う必要があると述べている.
