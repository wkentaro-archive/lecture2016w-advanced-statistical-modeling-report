\documentclass[letterpaper, 12pt, onecolumn, conference]{ieeeconf}  % Comment this line out if you need a4paper
%\documentclass[a4paper, 10pt, conference]{ieeeconf}      % Use this line for a4 paper

\IEEEoverridecommandlockouts                              % This command is only needed if 
                                                          % you want to use the \thanks command

\overrideIEEEmargins                                      % Needed to meet printer requirements.

% See the \addtolength command later in the file to balance the column lengths
% on the last page of the document

\usepackage[dvipdfmx]{graphicx}
\graphicspath{{figs/}}
\usepackage[position=bottom]{subfig}
\usepackage{amsmath}
\usepackage{amssymb}
\usepackage{algorithm}
\usepackage{algpseudocode}
\usepackage{times}
%
\newcommand{\figref}[1]{Fig.\ref{figure:#1}}
\newcommand{\tabref}[1]{Table \ref{table:#1}}
\newcommand{\argmax}{\operatornamewithlimits{argmax}}
\newcommand{\argmin}{\operatornamewithlimits{argmin}}


\title{\LARGE \bf
  Report about paper \\ ``Universal Correspondence Network''
  \\
  \large 先端統計モデリング論, 2016A
}


\author{和田健太郎, 48-166636% <-this % stops a space
% \thanks{*This work was not supported by any organization}% <-this % stops a space
% \thanks{$^{1}$Albert Author is with Faculty of Electrical Engineering, Mathematics and Computer Science,
%         University of Twente, 7500 AE Enschede, The Netherlands
%         {\tt\small albert.author@papercept.net}}%
% \thanks{$^{2}$Bernard D. Researcheris with the Department of Electrical Engineering, Wright State University,
%         Dayton, OH 45435, USA
%         {\tt\small b.d.researcher@ieee.org}}%
}


\begin{document}

\maketitle
\thispagestyle{empty}
\pagestyle{empty}

\input src/abst_report.tex
\input src/abst.tex
\input src/why_selected.tex
\input src/pros_cons.tex
\input src/construction.tex
\input src/way_to_write_paper.tex
\input src/difference_table.tex
\input src/detail.tex

\addtolength{\textheight}{-12cm}   % This command serves to balance the column lengths
                                  % on the last page of the document manually. It shortens
                                  % the textheight of the last page by a suitable amount.
                                  % This command does not take effect until the next page
                                  % so it should come on the page before the last. Make
                                  % sure that you do not shorten the textheight too much.

\bibliographystyle{junsrt}
\bibliography{main}

\end{document}
